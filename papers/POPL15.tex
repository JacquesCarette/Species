%% -*- mode: LaTeX; compile-command: "mk" -*-

\documentclass[preprint,authoryear]{sigplanconf}

\usepackage{amsmath}

\begin{document}

\special{papersize=8.5in,11in}
\setlength{\pdfpageheight}{\paperheight}
\setlength{\pdfpagewidth}{\paperwidth}

\conferenceinfo{POPL '15}{January 11--18, 2015, Mumbai, India}
\copyrightyear{2015}
%\copyrightdata{978-1-nnnn-nnnn-n/yy/mm}
%\doi{nnnnnnn.nnnnnnn}

% Uncomment one of the following two, if you are not going for the
% traditional copyright transfer agreement.

\exclusivelicense                % ACM gets exclusive license to publish,
                                 % you retain copyright

%\permissiontopublish             % ACM gets nonexclusive license to publish
                                  % (paid open-access papers,
                                  % short abstracts)

\titlebanner{DRAFT --- do not distribute}        % These are ignored unless
\preprintfooter{Submitted to POPL'15}   % 'preprint' option specified.

\title{Type-Theoretic Foundations for Combinatorial Species}
%\subtitle{Subtitle Text, if any}

\authorinfo{Brent A. Yorgey \\ Stephanie Weirich}
{Dept. of Computer and Information Science\\ The University of Pennsylvania\\
Philadelphia, Pennsylvania, USA}
{\{byorgey,sweirich\}@cis.upenn.edu}

\authorinfo{Jacques Carette}
{Dept. of Computing and Software\\ McMaster University\\
Hamilton, Ontario, Canada}
{carette@mcmaster.ca}

\maketitle

\begin{abstract}
  This paper develops a constructive definition of Joyal's theory of
  combinatorial species using homotopy type theory. We justify our definitions
  by generalizing various operations on species to arbitrary functor
  categories. In particular, we lift monoidal structures from the codomain
  category to define species sum and
  Cartesian product, and Day convolution to push monoidal structures from
  the domain category to define partitional and arithmetic
  products. This foundational work is the first step in the application of the
  theory of species to a wide class of data structures.
\end{abstract}

\category{CR-number}{subcategory}{third-level}

% general terms are not compulsory anymore,
% you may leave them out
% \terms
% term1, term2

\keywords
Combinatorial Species, Homotopy Type Theory

\section{Introduction}


\acks

Acknowledgments, if needed.

% We recommend abbrvnat bibliography style.

\bibliographystyle{abbrvnat}

% The bibliography should be embedded for final submission.

% \begin{thebibliography}{}
% \softraggedright

% \bibitem[Smith et~al.(2009)Smith, Jones]{smith02}
% P. Q. Smith, and X. Y. Jones. ...reference text...

% \end{thebibliography}


\end{document}
