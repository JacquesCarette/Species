\documentclass[9pt]{sigplanconf}

%\usepackage{latex8}
%\usepackage{times}
\usepackage{amsmath}
\usepackage{latexsym}
\usepackage{amssymb}
\usepackage{proof}
\usepackage{comment}
\usepackage{url}
\usepackage{xspace}

\pdfpagewidth=8.5in
\pdfpageheight=11in

\newcommand{\lam}[2]{\lambda #1 . #2}

\newcommand{\rase}[1]{\ulcorner #1 \urcorner}
\newcommand{\lowr}[1]{\llcorner #1 \lrcorner}

\newtheorem{theorem}{Theorem}
\newtheorem{proposition}{Proposition}
\newtheorem{definition}{Definition}
\newtheorem{lemma}{Lemma}

\begin{document}

\title{Towards Typing for Small-Step Direct Reflection}

\authorinfo{Jacques Carette}
{Dept. of Computing and Software\\ McMaster University\\
Hamilton, Ontario, Canada}
{carette@mcmaster.ca}

\authorinfo{Brent Yorgey}
{Dept. of Computer and Information Sciences\\ The University of Pennsylvania\\
Philadelphia, Pennsylvania, USA}
{byorgey@cis.upenn.edu}

\maketitle

\begin{abstract}
\end{abstract}

\category{D.3.2}{Programming Languages}{Applicative (functional) languages}
\category{F.3.1}{Logics and Meanings of Programs}{Specifying and Verifying and Reasoning about Programs}

\terms
Languages, Types

\section{Introduction}\label{sec:intro}

\section{Species as semantics}\label{sec:specsem}
\begin{itemize}
\item brief recap of definition of species, categorically
\item short intro to operators on species
\item implicit species theorem
\end{itemize}

\section{A language of type constructors based on species}\label{sec:language}
\begin{itemize}
\item need declaration language, based on above
\end{itemize}

\section{Programming with species, part I}\label{sec:prog1}
\begin{itemize}
\item basic term language (introduction, elimination forms)
\item semantics, and usual theorems
\end{itemize}

\section{Programming with species, part II}\label{sec:prog2}
\begin{itemize}
\item polytypic/generic programming (includes map, fold, unfold, etc.)
\end{itemize}

\section{Species and Haskell}\label{sec:haskell}
\begin{itemize}
\item why the implicit species theorem's assumptions are always satisfied in Haskell (but not in traditional type theories)
\end{itemize}

\section{Related Work}\label{sec:related}
\begin{itemize}
\item containers, naturally 
\end{itemize}

\section{Conclusion}\label{sec:conclusion}

%\bibliographystyle{plainnat}
%\bibliography{paper}

\end{document}
